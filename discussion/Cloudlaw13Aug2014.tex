% ***********************************************
% Main document file
% for Modelling TiMo using RL
% Started: Nov 2011
% Updated: 20/04/2012

\documentclass[runningheads,a4paper]{llncs}

\usepackage{graphicx}
\usepackage{float}

\usepackage[cmex10]{amsmath}
\usepackage{array}
\usepackage{mdwmath}
\usepackage{mdwtab}
\usepackage{url}

\title{Open Issues on the Observance of Cloud Computing Law: A Survey}

% a short form should be given in case it is too long for the running head
\titlerunning{Cloud Computing Law}

%%\author{Carlos Molina--Jimenez\inst{1} \and Ioannis Sfyrakis\inst{2}}
%%\author{Carlos Molina--Jimenez\inst{1} \and Thomas \inst{1}}
\author{ Alice and Bob}


\institute{Computer laboratory, University of Cambridge,
        \email{carlos.molina@cl.cam.ac.uk}
}

\begin{document}

\maketitle

\begin{abstract}
Cloud Computing is 

\end{abstract}

%organizations or are independent developers that 
%volunteer their time to work on an open source 
%project.

\section{Introduction}
\label{Introduction}
The term \emph{cloud computing} became a buzzword around 2006 when Amazon and
Google and other companies started using to market their online computational
services. Unfortunately, technically, the term is not descriptive or intuitive;
thus it is frequently abused and misused. As discussed in~\cite{Vaquero2009}
cloud computing means different things to different people, all depending on
the aspect (for example, ubiquity, transparency, virtualization, 
business model, etc.) they wish to emphasise. An extensive list of definitions
of cloud computing is presented in \cite{LeeBadger2011}. 

In this paper we will 
define cloud computing as a paradigm that enables cloud subscribers
(also called clients, customers and users) network, on--demand and transparent 
access to a shared pool 
of computing resources of different levels of abstractions provided by other 
parties called the cloud providers.

The focus of our interest are public clouds (as opposite to private
clouds), that is clouds available to the general public for free
or in return of a fee. Thus from here on, we will refer to
public cloud providers, simply as cloud providers.

Central to this definition and to the focus of this paper is the
concept of \emph{transparency}. At this point we shall only provide and 
intuitive explanation of this concept and mention that  transparency
means seamless access to computating resources. Transparency conceal
from the client the complexity of the software and harware infrastructure
used by the cloud provider to deliver the service. Thus the client 
gets the illusion of an endless availability of ubiquitous resources 
whose actual realizations, physical locations and administration 
are not necessarily under his control or visible to him---all depending on 
the level of abstraction of the cloud resource used.


Transparency is certainly a useful feature of cloud computing. However
its is also a source of several concerns related to data security, privacy
and regulatory compliance. For instance, the Cloud Security Guidance
published by the UK Government~\cite{CSprinciples2014} states that
\emph{It is important that the locations in which consumer data is stored, 
      processed and managed from \textbf{are known} as organisations will need to 
      understand the legal circumstances in which their data could be accessed 
      without their consent.}
 
The aim of this paper is to investigate the technical aspect
of these issues and present a state of the art discussion of the available
mechanisms for specifying and enforcing data security regulations, and 
for resolving potential conflicts between the parties involved. In other
words, we would like to identify what can be solved programmatically
(say with current technology) computational means and what needs to be left for
human judgement. 

Regarding the level of abstraction of the resources is it worth mentioning 
that the resource made available to consumers
range from providing basic computational resources such
as network communication, storage and compute power (infrastructure as a service,
IaaS) to sophisticate enterprise application services (soft-
ware as a service SaaS).  Consequently some authors use the 
term \emph{cloud services} as synonymous to cloud resources.
An in--depth discussion of these levels of abstractions
can be found in~\cite{LeeBadger2011,MichaelArmbrust2009}. 

Regarding accounting of resource consumption, it is worth clarifying
that a common business model is to charge consumers on a 
pay--per--use basis where they periodically pay for the resources 
they have consumed. This business model is currently 
in use by Amazon AWS and other leading cloud providers. 
An alternative and widely used model is the \emph{free of charge}
model used my Google gmail, Yahoo email, FaceBook and other
companies that provide free of charge services in return
for collecting data about customers and displaying 
advertisements. As discussed in~\cite{leontiadis2012}, this 
revenue model is the basis for the "free cloud" services
that many users rely on.  Another
possible alternative that has been suggested by supporters
of good causes is free cloud services to citizens, that is,
paid from public funds. 



\section{Law and Cloud Data Protection}
In our definition of subscriber cloud data we include 
both application and administration data. By application data
we mean data used or produced (either directly or as
by--product) by the applications deployed
by the subscribers. In the same order, by administration data
we understand data collected by cloud providers about their
subscribers for administration and regulatory purposes.

Similarly to data transmitted, stored and processed by
conventional computation means, cloud data is subject
to legal regulations  that depend on several factors
such as the nature of the data and its physical location.

A detail description of the different categories of data and
the legislations that apply to them fall out of the scope
of our work. Our aim is to examine whether  it is 
possible to guarantee the observance of the legislations
of interest with current technology.  


For example, the Consumer Data Privacy in a Networked World~\cite{CDP2012}
states that \emph{Consumers have the right to reasonable limits on the 
personal data that companies collect and retain}. The problem with
this statement it that it is hard to express it programmatically so that
it can be mechanically manipulated. For instance, the meaning of
"reasonable" is subject to personal interpretation. 

%% In addition, the terms "collect and retain" need further 
%% clarification. In general we believe that terms like 
%% what a company can see (in plain text), retain (store), use (for other purposes)
%% share with other companies for other purposes. ... need to be 
%% clarified. I beleive that these issues are related but independent from
%% each other and that they can be addressed independently by
%% different techn9ologies.
%%


\section{Deployment of Cloud Applications}
We envisage cloud applications of arbitrary complexity including  
scientific computations (for example, scientific workflows) that demand large amount 
of computational resources~\cite{Vecchiola2009,Deelman2008} and
collaborative applications that involve several independent
parties that share resources (for example, databases) but
does not necessarily trust each other.
Good examples of collaborative applications that highlight the
importance of data legislation are cross--organizational
business processes that involve the participation of several business 
partners such as car--rental~\cite{Aalst2011} and conference management systems like
EasyChair~\cite{Ryan2011}.

The deployment of complex applications demands the deployment of
several cloud components (virtual machines, storage, databases, etc.) that
the customer requests from a single of several cloud providers. This
idea is illustrated in Fig.~\ref{AppDeployment}.

In the figure $C_A$, $C_B$ and $C_C$ represent customers interested in
deploying, respectively, their applications $A_A$, $A_B$ and $A_C$ within
cloud providers, ${CP}_A$, ${CP}_B$ and ${CP}_C$, respectively. Application
In this example, $A_A$ is shown unfolded and consists of five components  
$C_1$, $C_2$, $C_3$, $C_4$ and $C_5$. 

The double arrowed lines represent communication channels. They 
suggest that the three applications interact with each other,
presumably to execute a cross--organisational business process.
Central to our discussion are the communication lines between
the customers and their providers with T\&C (Terms and Conditions)
lables. We assume that at run time, customers will interact with
their cloud providers to deploy, monitor, tune and undeploy
his components. 


The terms and conditions are legal commitments (contract) agreed
and signed between customers and providers. In brief, they
stipulate how the cloud service is expected to be delivered
by the cloud provider and used by the customer. In other words,
they stipulated what actions the parties (customer and cloud provider)
has the right, obligation or prohibition to execute during the
contractual time in order to observe the expected behaviour
of the service. For example, the terms and conditions might
dictate that the customer has the right to instantiate up
to 64 virtual machines and the obligation to pay the incurred
bill by a certain date. In the same order, they might stipulate
that the cloud provider is obliged to recover the service within
24 hrs after crashes. Equally important, terms and conditions
include clauses related to data security, privacy and 
regulatory compliance. In practice, they stipulate what national
and international regulations they honour.  For example,
they might referer to national regulations such as
Data Protection Act 1998 of the UK~\cite{DPAGuide2014} or the 
Consumer Privacy Bill of Right of the US~\cite{CDP2012} and 
the EU Data Protection Directive~\cite{DPDoct1995}.
  

For example, regarding data location, the terms and conditions 
expressed by Microsoft in its Azure web page~\cite{MSAzurePrivacy2014} 
stipulate that \emph{Microsoft may transfer Customer Data within 
a major geographic region (for example, within the United States 
or within Europe) for data redundancy or other purposes}. Similarly,
the terms and conditions published by Amazon AWS regarding their
S3 storage~\cite{AmazonS3faqs2014} stipulate that
\emph{Amazon S3 offers storage in the US Standard, US West (Oregon), 
      US West (Northern California), EU (Ireland), Asia Pacific (Singapore), 
      Asia Pacific (Tokyo), Asia Pacific (Sydney), South America (Sao Paulo), 
      and AWS GovCloud (US) regions. You specify a region when you create 
      your Amazon S3 bucket. Within that region, your objects are redundantly 
      stored on multiple devices across multiple facilities.} In fact,
Amazon AWS offer means for its customers to choose the geographical
location of his components at allocation time. Thus a customer is in a position
to decide at deployment time, the geographical location (US, Europe, 
Asia Pacific, etc.) of S3 storage expected to store personal data. 

At first glance, from the above disccussion it seems that the cloud computing 
community has all the elements (legislations and technical means) needed to 
enforce data protection in the cloud. We believe that this 
observation holds but only for simple applications, such as those that involve one of two 
cloud components colocated within a single cloud provider. However, we 
argue that the problem is of \emph{scalability}. We will explain the challenges
we have identified that impact the enforcement of data protection in the 
cloud. Our view is that the enforcement of regulations of
cloud data at large scale can be achieved only with the assistant of
programatic means. We consider that manual approaches like Amazon's 
facilities that allow customers to choose geographycal regions at
deployment time, are insatisfactory solution because they no not
scale well. 

\subsection{Legislations}
To be effectively used in practice, cloud computing regulations, such as the Data Protection
Directive~\cite{DPDoct1995}, need to be enforced and monitored for detecting potential violations
at run time or at off--line log examination. This implies that the textual description
of the regulations need to be mapped onto precise notation that is amenable to
computer manipulation.  The main difficulty that technical people face when presented 
with this tasks is the gap between the legal and technical domains. Legal documents 
are written at targeted at humans that are expected to use their judgement to 
make sense of them.  Consequently, they normally contain ambiguities and subjective
terminoloy and suffer from omissions and conflicts.  As an example, let us
take the first princliple of article 25 of the Data Protection Directive which is related to the transfer
of personal data to third countries:

\emph{ 1. The Member States shall provide that the transfer to a third country of 
       personal data which are undergoing processing or are intended for processing 
       after transfer may take place only if, without prejudice to compliance with 
       the national provisions adopted pursuant to the other provisions of this 
       Directive, the third country in question ensures an adequate level of 
       protection.}

The \emph{adequate} is hard to express in computer notation and can only be
interpreted by human judgement. As a second example, let us examine 
point c) of Article 6 which states that:
\emph{Member States shall provide that personal data must be:
adequate, relevant and not excessive in relation to the purposes for 
which they are collected and/or further processed}. Again, 
the terms \emph{adequate, relevant and not excessive} require human
judgement for their interpretation.
In summary, the research challenge here is to identify what statements can
be capture in computer notation and what needs to be left for humans to
interpret.

\subsection{Deployment with the observance of obligations}
At run time, the broker will regularly estimate the cost of the service by
monitoring the actual resource usage and dynamically perform any changes such as expand (shrink)
resource pool, switch between providers and so forth. Furthermore, billing and service usage information is
presented by the broker to the customer in a manner that enables the customer to relate this information to
their business goals, making the task of revising and re-negotiating service usage policies with the broker
straightforward. Such re-negotiation can happen dynamically, as the user needs change.

Cloud applications need configuration in order to be regulation
compliance--the designer needs to ensura that execution paths
that compromise the regulation are excluded.

service broker implementing the facility of smart metering for the consumer by
translating consumer’s requirements into an adequately provisioned value-added service, that is mapped
onto one or more cloud services. 

At run time, the broker will regularly estimate the cost of the service by
monitoring the actual resource usage and dynamically perform any changes such as expand (shrink)
resource pool, switch between providers and so forth. Furthermore, billing and service usage information is
presented by the broker to the customer in a manner that enables the customer to relate this information to
their business goals, making the task of revising and re-negotiating service usage policies with the broker
straightforward. Such re-negotiation can happen dynamically, as the user needs change.


\subsection{Consumer's awareness of data regulations}
We expect that a typical cloud customer will dynamically expand 
and shrink his resource pool and possibly switch between 
providers, as needed to accomplish his business goals.
In response, the provider reacts to his customer requests by executing 
the low level technical procedures (for example, instantiation of a new
virtual machine or allocation of a volume of storage) to realise the 
requests, under the observance of the terms and conditions 
agreed with the customer including those related to data
regulations. For instance, the provider will not replicate
databases outside the EU region if the terms and conditions
agreed with the customer estipulate so.

However, the cloud provider can do nothing to prevent the 
customer from accidentally (or deliberately) violating 
data regulations. The difficulty is that as explained
in~\cite{Winkler2011} regulations apply to specific data. 
Let us take medical records as an example.
In the US the Health Insurance Portability and Accountability 
(HIPAA) applies to medical records. Thus
customers that collect and store medical records about their employees
are at risk of inadvertesly uploading those records to the cloud in 
readble format. The risk of making this kind of mistakes
magnifies with cloud applications with scores of components.


It appears cloud data protection is a colaborative commitment between 
the cloud provider and the customer but demands a customer's
clare view of what data he is transfering, processing and storing
and what cloud components he is deploying or undeploying. In other 
to make an informed decision about the deployment (or undeployment) of 
a given component, the customer needs to have a precise
view of the current state of his application.

As argued earlier, cloud applications and regulations might be 
complex. The fact that cloud providers lease individual components
under specific terms and conditions (for example, Amazon S3
buckets come with their own terms and conditions and so
do volumes of Elastic Block Store and EC2 instances) add to 
this complexity.

Thus we argue in favour of formal models that assist 
cloud customers to deploy their applications. Such models should
help customers systematically and mechanically reason (for example at
run time or predeployment time) about the
behaviour of their application in terms of data protection so
that they can assess the risk of data regulation violations.
For example, a model can indicate that data $D$ cannot
be transfered along path $P$ before it has been proccessed
by activity $A$---a formal expression for requesting that
personal data cannot travel outside EU region unless it has
been encrypted. 
 
\subsection{Formal models for automatic compliance checking}
Formal models for automatic checking of regulatory compliance have been
suggested in the literature. A good introduction into
the topic is~\cite{Becker2012} where the use of graph--based
models is suggested. 
The deployment of scientific application in the cloud
under strict privacy constrainst is dicussed in~\cite{Watson2011}.
The author argues that task allocation becomes a challenging problem when
the data that some of the tasks input or output is subject
to security requirements such as privacy.  In these situations, 
the workflow designer needs to examine the security guarantees offered
by providers to ensure that his security requirements are
not violated during task execution, data storage or data transfer between
different cloud providers. For instance,  the result of the
analysis might dictate that some tasks can be executed only
within the local cloud. The author suggests a systematic method for workflow
partition with the observance of security requirements based 
on the Bell--LaPadula multi--level access control 
model. 
 

\section{Data Centers Management and Compliance}



\section{Related Work}

\cite{Vaquero2011}

\cite{Millard2013}

\cite{Mitrakas2013}

\section*{Acknowledgment}


%%References
\bibliographystyle{splncs}
\bibliography{bibliography}

\end{document}


%% howpublished = "\url{http://...}"


%% [2] “Generalizability and Applicability of Model--Based Business Process Compliance--Checking 
%% Approaches---A State--of--The--Art Analysis and Research Roadmap”, 
%% Jorg Becker, Patrick Delfmann, Mathias Eggert and Sebastian Schwittay, BuR Business 
%% Research Journal, v5, n2, Nov 2012.


ll these concepts have been widely dicussed in the literature  
(see for example \cite{LeeBadger2011}), yet  for the sake
of completness and to prevent misunderstandings we will briefly summarise them
next.

\emph{cloud subscribers:} a subscriber is the party (a human being or an organization) that
     uses the cloud. In current literature, the terms subscriber, customer, user and
     client are used as synonimous.
\emph{client application:} a software application that a subscriber uses to
     access the cloud computing resources remotely and is normally deployed on
     his desktop computer, laptop or mobile phone.


 We justify this choice on the basis that this definition emphasises
the governace of cloud computing--the central topic of this paper.





