% ***********************************************
% Main document file
% for Modelling TiMo using RL
% Started: Nov 2011
% Updated: 20/04/2012

\documentclass[runningheads,a4paper]{llncs}

\usepackage{graphicx}
\usepackage{float}

\usepackage[cmex10]{amsmath}
\usepackage{array}
\usepackage{mdwmath}
\usepackage{mdwtab}
\usepackage{url}

\title{Contacts Application Demo (13 mar 2015)}

% a short form should be given in case it is too long for the running head
\titlerunning{Explanotary remarks.}

%%\author{Carlos Molina--Jimenez\inst{1} \and Ioannis Sfyrakis\inst{2}}
%%\author{Carlos Molina--Jimenez\inst{1} \and Thomas \inst{1}}
\author{ Alice and Bob}


\institute{Computer laboratory, University of Cambridge,
        \email{carlos.molina@cl.cam.ac.uk}
}

\begin{document}
\maketitle

\begin{abstract}
The moana layer was implemented by Yan. 
\end{abstract}

%organizations or are independent developers that 
%volunteer their time to work on an open source 
%project.

\section{Contact data is modelled as tuples}
\label{Introduction}

\subsection{Data about Jon}
\begin{verbatim}
  
let jon =
  Helper.to_tuple_lst
    "{(a,fn,Jon,contacts)
          (a,last,Crowcroft,contacts)    
   (a,email,jon.crowcroft@cl.cam.ac.uk,contacts)
         (a,twitter,@tforcworc,contacts)
         (a,mobile, 617-000-0001,contacts)
   (a,title,Professor,contacts) 
         (a,image,jon.jpg, contacts)
         (a,knows,c,contacts)
         (a,knows,d,contacts)
         (a,knows,e,contacts)
        }"
\end{verbatim}

\subsection{Every line is a tuple}
\begin{verbatim}
let tup_jo1=    (* tuple of the fisrt line *)
 {
  subj=  Constant "a";          (* a is a unique id *)
  pred=  Constant "fn";         (* fn= first name   *)
  obj=   Constant "Jon";
  ctxt=  Constant "contacts";
  time_stp= None;
  sign= None;
 }
 (* In Eng: "a has_fn Jon" *)



let tup_jo2=  (* tuple of the second line *)
 {
  subj=  Constant "a";
  pred=  Constant "last";        (* last: last name *)
  obj=   Constant "Crowcroft";
  ctxt=  Constant "contacts";
  time_stp= None;
  sign= None;
 }
 (* In Eng: "a has_last Crowcroft" *)


let tup_jo3=
 {
 subj= Constant "a";
 pred= Constant "email";
 obj=  Constant "jon.crowcroft@cl.cam.ac.uk";
 ctxt= Constant "contacts";
 time_stp= None;
 sign= None;
 }
 (* In Eng: "a has_email Crowcroft" *)


let tup_jo5=
 {
  subj= Constant "a";
  pred= Constant  "mobile"; 
  obj=  Constant  "617-000-0001";
  ctxt= Constant  "contacts";
  time_stp= None;
  sign= None;
 }

...
let tup_jo10=  (* tuple of the last line *)
 {
  subj= Constant "a";
  pred= Constant "knows";
  obj=  Constant "e";
  ctxt= Constant "contacts";
  time_stp= None;
  sign= None;
 }
\end{verbatim}



\subsection{More examples of contacs data}  

\begin{verbatim}
let amir =
  Helper.to_tuple_lst
    "{(b,fn,Amir, contacts)
                (b,last,Chaudhry,contacts)    
   (b,email,amir.chaudhry@cl.cam.ac.uk,contacts)
         (b,twitter,@amirmc,contacts)
         (b,image,amir.jpg, contacts)
   (b,title,Postdoc,contacts)
   (b,knows,a,contacts)
   (b,knows,c,contacts)
         (b,knows,d,contacts)
        }"
  
let anil =
  Helper.to_tuple_lst
    "{(c,fn,Anil, contacts)
          (c,last,Madhavapeddy, contacts)
   (c,email,anil@recoil.org,contacts)    
         (c,image,anil.jpg, contacts)
         (c,twitter,@avsm,contacts) 
         (c,email,anil.madhavapeddy@recoil.org,contacts)
   (c,title,Lecturer,contacts)}"
  
let carlos =
  Helper.to_tuple_lst
    "{(d,fn,Carlos, contacts)
         (d,last,Molina-jimenez, contacts)
         (d,image,carlos.jpg, contacts)    
   (d,email,cm770@cam.ac.uk,contacts)        
   (d,title,Postdoc,contacts)
         (d,twitter,@carlos,contacts) 
         (d,knows,a,contacts)
         (d,knows,c,contacts)}"
  
let richard =
  Helper.to_tuple_lst
    "{(e,fn,Richard,contacts)    
                (e,last,Mortier,contacts)
   (e,email,richard.mortier@nottingham.ac.uk,contacts)
         (e,image,mort.png, contacts)
         (e,twitter,@mort__,contacts)        
   (e,title,Lecturer,contacts)
         (e,knows,c,contacts)
         (e,knows,d,contacts)
        }"
\end{verbatim}

\subsection{SPARQL queries can be placed against moana}
SPARQL queries can be placed against moana to retrive information from irmim. The response
is the set (possibly empty) of tuples that satisfy the query.

\begin{verbatim}
let qt1jon =
 {
  subj= Variable "a";        (* a: unique id for Jon *)
  pred= Constant "knows";
  obj=  Constant "?y";
  ctxt= Constant "contacts";
  time_stp= None;
  sign= None;
 }
(* In Eng: retrieve all the people that Jon knows *)

let qt1jon =
 {
  subj= Variable "a";        (* a: unique id used for Jon *)
  pred= Constant "knows";
  obj=  Constant "?y";
  ctxt= Constant "contacts";
  time_stp= None;
  sign= None;
 }
(* In Eng: retrieve all the people that Jon knows *)


let qt1amir =
 {
  subj= Variable "b";       (* b: unique id used for amir *)
  pred= Constant "knows";
  obj=  Constant "?y";
  ctxt= Constant "contacts";
  time_stp= None;
  sign= None;
 }
(* In Eng: retrieve all the people that amir knows *)
\end{verbatim}

\section{Policies can be used to constrain visibility of contacts data}

\begin{verbatim}
let q1_policy_email_and_fn =
  "MAP {
         b,  canView, ?x,     policies
         ?x, knows,   ?o,     contacts
         ?o, email,   ?email, contacts
         ?o, fn,      ?n,     contacts
        }"
(* In Eng: Bring all tuples that "b" can view. 
           e,g., if "b" can view "a" then, bring 
           everyone whom "a" knows, their "email" and 
           "fn" name--but nothing else.
           A query to retrive the "last" name of a person will 
           produce no tuples.

let q1_policy_email_fn_and_last =
  "MAP {
         b,  canView, ?x,     policies
         ?x, knows,   ?o,     contacts
         ?o, email,   ?email, contacts
         ?o, fn,      ?n,     contacts
         ?o, last,    ?last,  contacts
        }"
(* In Eng: Bring all tuples that "b" can view. 
           e,g., if "b" can view "a" then, bring 
           everyone whom "a" knows, their "email", 
           "fn" and "last" name--but nothing else.
           A query to retrive the "last" name of a person will 
           produce some tuple.
\end{verbatim}


\bibliographystyle{splncs}
\bibliography{bibliography}
\end{document}





